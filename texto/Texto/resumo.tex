% Não alterar esse arquivo
\label{resumo}
\begin{resumo}
%\resumoTrabalho

%\Contexto e objetivo

Emoções são uma das principais características que nos tornam humanos. Algo abstrato, deduzido e catalogado por nós mesmos, que dita toda manifestação da percepção humana no mundo. Um dos grandes desafios atuais da ciência é conseguir, apropriadamente, interceptar e interpretar de forma coesa esses sinais neurais que constituem as emoções. Contudo, após este desafio ser alcançado, o próximo seria como e onde usar essas novas possíveis entradas e variáveis dentro de sistemas digitais. Tendo isso em vista, foi realizado nesse trabalho a interpretação de sinais neurológicos pré-classificados em um sistema de aprendizado de máquina executado ao mesmo tempo que um jogo de computador, sendo este jogo um utilizador da saída destes dados para variar o contexto em que o jogador se encontra, através de um loop de feedback entre o jogador e o jogo. Os métodos utilizados neste trabalho são formados pela interpretação de sinais neurais através de inteligência artificial e detecção de padrões, desenvolvendo novas e melhorando as já existentes metodologias, estratégias e formas de imersão e retenção de público utilizadas na industria de entretenimento voltada para jogos digitais. Também foi utilizado predição de escolhas e emoções através de dados estatísticos de jogabilidades e decisões preexistentes\todo{rever ao desenrolar do projeto}. A proveniência dos sinais neurais se da pelo uso de bases de dados públicas que contemplam detalhes de atividade cerebral de voluntários enquanto tarefas específicas eram desenvolvidas, capturados através de EEG (eletroencefalografia) e outros métodos. Estes e outros métodos e materiais são mais detalhadamente descritos nas próximas sessões deste trabalho.

\palavraschave{\palavraChaveUm; \palavraChaveDois; \palavraChaveTres;  \palavraChaveQuatro; \palavraChaveCinco; \palavraChaveSeis}
\end{resumo}