\chapter{RESULTADOS ESPERADOS} \label{resultados}

%Baseando-se na metologia proposta e no conteúdo apresentado nos capítulos \ref{trabs} e \ref{conceitos}, este trabalho tem a pretenção de obter um sistema preditivo com boa qualidade quando comparado com outros trabalhos apresentados.

%Primeiramente, espera-se uma melhoria na precisão das predições e uma diminuição no custo computacional para gerá-las utilizando-se os modelos baseados em clusterização propostos por este trabalho em comparação com filtros colaborativos tradicionais.

%Em seguida, espera-se que o sistema híbrido proposto atinja uma precisão melhor do que a obtida por outros trabalhos que possuem sistemas também baseados em clusterização.

%Por fim, este trabalho visa comparar os resultados obtidos pelo sistema proposto com os resultados de trabalhos que utilizam outras técnicas tanto no âmbito de precisão das predições como em custo computacional para construção do modelo e para geração de predições.

Levando-se em conta o que foi observado na metodologia proposta deste trabalho, é esperado que seja obtido um modo de clusterização, onde a precisão dos filtros colaborativos supere o desempenho das técnicas encontradas atualmente.

Com isso, os sistemas de distribuição que utilizarem essas técnicas aplicadas à sua base de dados conseguirão realizar melhores predições, pois os dados em questão estarão relacionados de forma a aumentar a eficiência do sistema.
Além disso, a melhora no desempenho computacional permitirá menores custos monetários com hardware ou melhores resultados mantendo-se os mesmos recursos computacionais.

Pela observação dos aspectos analisados, esse trabalho melhorará a utilização de sistemas de recomendação. Consequentemente, trazendo benefícios aos usuários e às empresas responsáveis por esses sistemas.