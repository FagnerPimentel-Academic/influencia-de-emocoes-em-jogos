\chapter{INTRODUÇÃO}
\label{intro}

Não é de hoje que um dos principais fatores para o avanço da neurociência foram os métodos de detecção de ondas cerebrais \cite{eegimportanciasite}. Através deles, se tornou confiável capturar e interpretar sinais neurais em animais e humanos. No que tange o emocional, diversos estudos foram e ainda são realizados para que dados obtidos por métodos como Eletroencefalograma sejam interpretados de maneira fidedigna.

Neste trabalho, esses dados foram utilizados em um plano de fundo que nos ajude a entender as possibilidades de interação entre a saída desses sinais e um sistema computacional. Este plano de fundo, como destacado, será um jogo de computador que reage aos estados emocionais do jogador dentro de seu ecossistema através de um banco de dados já recolhido para a análise desses sinais.

\section{Contextualização}

A área de pesquisa de captação neural para interpretação desses dados evoluiu muito desde o surgimento de técnicas como o Eletroencefalograma \cite{rocha2022analise}. A partir destes métodos, foram desenvolvidas e maturadas diversas técnicas para que esses dados fossem interpretados em sua completude. Isto é, o que a principio era utilizado apenas para saber níveis de estresse ou relaxamento, devido o conhecimento ainda muito vago na interpretação dos sinais, hoje é utilizado em estudos até mesmo na reconstrução de imagens conjuradas na imaginação \cite{shimizu2022improving}. Para captar estes sinais, métodos como Eletroencefalograma (EEG) podem ser utilizados \cite{moises2020reconhecimento}, bem como outros menos sofisticados, como captação de sinais de ansiedade e estresse para classificar emoções positivas e negativas \cite{nalepa2019emotionalcontext}. Independente dos métodos utilizados, este trabalho transcorre utilizando bases de dados prontas e classificadas como forma de treinamento e entrada num sistema de classificação de padrões. Esse sistema irá, então, se comunicar através de uma interface com um jogo, afim de demonstrar a efetividade e desempenho do algoritmo de interpretação, bem como os impactos de um sistema como esse na jogabilidade e imersão do jogador.

Por sua vez, a área dos jogos digitais é bastante ampla na aplicação de conceitos de programação, inteligência artificial e arte. A maior parte dos jogos incluem em suas experiências algum nível de adaptabilidade com foco no jogador: seletores de dificuldade, filtros de cor para pessoas com daltonismo e até mesmo compatibilidade com controles especiais. Essa capacidade de adaptação tem por finalidade tornar a experiência do usuário mais confortável, mesmo se o jogador tiver algum tipo de deficiência. Esse tipo de adaptabilidade é também chamado de acessibilidade \cite{acessibilidadenosjogos}. 

Contudo, também existe a adaptabilidade que visa não o conforto do jogador, mas a capacidade do jogo de se adaptar a certas formas de interação e responder de formas diferentes a essas interações, gerando experiências diferentes. Essa é uma prática de mais complexa implementação, mas que ganha espaço devido a alta variação na experiência individual de cada jogador e a quantidade maior de possibilidades e caminhos que podem ser percorridos durante uma dessas experiências. Isso traz diferentes experiências para diferentes jogadores, tornando-as mais únicas a cada ambiente e situação.

No entanto, há poucos exemplos no mercado que chegaram ao ponto de capturar diretamente o que um jogador sente para então influenciar a jogabilidade em um loop de feedback: muitos jogos de terror usam o microfone, por exemplo, para monitorar gritos ou falas afim de atrair um monstro. O jogo Emovere \cite{filipa2019emovere} utiliza batimentos cardíacos para influenciar a mecânica de combate. O objetivo deste trabalho é construir um jogo mais próximo do segundo exemplo, utilizando sinais emocionais do jogador para influenciar o ambiente e mecânicas.

\section{Objetivo}

O objetivo deste trabalho é analisar, identificar e classificar determinados sinais neurais que caracterizem estados emocionais, afim de utilizá-los em um sistema final com ênfase na área de jogos digitais utilizando duas bases de dados publicas e outras bases dos trabalhos estudados mais a frente. Com isso, foram estudados os potenciais benefícios e malefícios da influência deste sistema dinâmico em relação a um jogo que não tenha esse sistema, analisando o quanto o jogador é influenciado e o quanto essas mudanças podem agregar a favor ou em detrimento do jogo.

\section{Estrutura do Trabalho}

O restante deste trabalho é dividido da seguinte maneira: No capítulo \ref{conceitos}, são apresentados os principais conceitos que serão utilizados, como Sistemas Sensíveis a Contexto, Emoções por Nível de Excitação e Valência, Loop de Biofeedback, Game Engine e outros. No capítulo \ref{trabs}, são apresentados os principais trabalhos relacionados pesquisados ao longo da elaboração deste trabalho, que é detalhado melhor no capítulo \ref{metodologia}, onde são apresentadas as ferramentas, métodos, materiais e métricas utilizadas para o estudo e elaboração do projeto, como a elaboração do algorítimo de padrões, simulação dentro do jogo construído, treinamento utilizando a base, verificação da eficiência do algorítimo, etc. \todo{completar com o restante dos capitulos}

